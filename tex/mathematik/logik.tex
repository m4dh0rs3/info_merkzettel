\chapter{Logik}

\section{Aussagenlogik}

\index{Aussage}
\paragraph{Aussage} Satz/Formel entweder wahr oder falsch;
,,-form`` bei zu wenig Infos.

\index{Theorem}
\paragraph{Theoreme} sind wahre Aussagen.

\subsection{Junktoren}

\begin{description}
  \item [Negation
        \index{Negation}
        $\boldsymbol{\neg} \mathcal{A}$]
        ,,Nicht``
        (\mintinline{Java}{!}, \mintinline{Lua}{~},
        \mzInline{
          \begin{circuitikz}
            \draw (0,0) node[not port] {};
          \end{circuitikz}
        })

  \item [Konjunkt.
        \index{Konjunktion}
        $\mathcal{A} \boldsymbol{\land} \mathcal{B}$]
        ,,und``
        (\mintinline{Java}{&&},
        \mzInline{
          \begin{circuitikz}
            \draw (0,0) node[and port] {};
          \end{circuitikz}
        })

  \item [Disjunkt.
        \index{Disjunktion}
        $\mathcal{A} \boldsymbol{\lor} \mathcal{B}$]
        ,,oder``
        (\mintinline{Java}{||},
        \mzInline{
          \begin{circuitikz}
            \draw (0,0) node[or port] {};
          \end{circuitikz}
        })

  \item [Implikat.
        \index{Implikation}
        $\mathcal{A} \boldsymbol{\Rightarrow} \mathcal{B}$]
        ,,Wenn, dann`` / \linebreak ,,$\mathcal{B}$``
        ($\rightarrow$, \mintinline{Java}{if})
        \begin{description}
          \item [$\mathcal{A} \Rightarrow \mathcal{B}$] ,,$\mathcal{A}$ hinreichend``
                \index{Hinreichende Bedingung}
          \item [$\mathcal{B} \Rightarrow \mathcal{A}$] ,,$\mathcal{A}$ notwendig``
                \index{Notwendige Bedingung}
        \end{description}

  \item [Äquiv.
        \index{Äquivalenz}
        $\mathcal{A} \boldsymbol{\Leftrightarrow} \mathcal{B}$]
        ,,Genau dann, wenn``
        ($\leftrightarrow$, $\equiv$,
        \mintinline{Java}{==},
        \mzInline{
          \begin{circuitikz}
            \draw (0,0) node[xnor port] {};
          \end{circuitikz}
        })
\end{description}

\index{Wahrheitswertetabelle}

\paragraph{Wahrheitswertetabelle} mit $2^n$ Zeilen für $n$ Atome. Konstruktionssystematik: Frequenz pro Atom verdoppeln.

\mzGraphic{
  \begin{tblr}{
    cells = {c},
    vline{3} = {-}{},
    hline{1-2,6} = {-}{},
      }
    $\mathcal{A}$ & $\mathcal{B}$ & $\neg \mathcal{A}$ & $\mathcal{A} \land \mathcal{B}$ & $\mathcal{A} \lor \mathcal{B}$ & $\mathcal{A} \Rightarrow \mathcal{B}$    & $\mathcal{A} \Leftrightarrow \mathcal{B}$ \\
    $0$ & $0$ & $1$      & $0$         & $0$        & $1$                  & $1$                   \\
    $0$ & $1$ & $1$      & $0$         & $1$        & $1$                  & $0$                   \\
    $1$ & $0$ & $0$      & $0$         & $1$        & $\mathbf{0}$ & $0$                   \\
    $1$ & $1$ & $0$      & $1$         & $1$        & $1$                  & $1$
  \end{tblr}
}

\mzBreak\

\begin{mzImportant}
  \mzGraphic{
    \begin{tblr}{
      cells = {c},
      cell{1}{1} = {c=2}{},
      cell{2}{3} = {r=2}{r},
      cell{4}{3} = {r=2}{r},
      cell{6}{3} = {r=2}{r},
      cell{8}{3} = {r=2}{r},
      cell{10}{3} = {r},
      cell{11}{3} = {r=2}{r},
      cell{13}{3} = {r=2}{r},
      cell{15}{3} = {r=3}{r},
      vline{2} = {2-17}{},
      hline{1-2,18} = {-}{},
        }
      \textbf{Äquivalente Formeln $\boldsymbol{\Leftrightarrow}$}               &                                                 & Bezeichnung \\
      $A \land B$                                & $B \land A$                                     & \index{Logische Kommutativität}Kommutativ  \\
      $A \lor B$                                 & $B \lor A$                                      &             \\
      $A \land (B \land C)$                      & $(A \land B) \land C$                           & \index{Logische Assoziativität}Assoziativ  \\
      $A \lor (B \lor C)$                        & $(A \lor B) \lor C$                             &             \\
      $A \land (B \lor C)$                       & $(A \land B) \lor (A \land C)$                  & \index{Logische Distributivität}Distributiv \\
      $A \lor (B \land C)$                       & $(A \lor B) \land (A \lor C)$                   &             \\
      $A \land A$                                & $A$                                             & \index{Idempotenz}Idempotenz  \\
      $A \lor A$                                 & $A$                                             &             \\
      $\neg \neg A$                              & $A$                                             & \index{Involution}Involution  \\
      $\neg (A \land B)$                         & $\neg A \boldsymbol{\lor} \neg B$  & \index{\textsc{De-Morgan}}\textsc{De-Morgan}   \\
      $\neg (A \lor B)$                          & $\neg A \boldsymbol{\land} \neg B$ &             \\
      $A \land (\mathbf{A} \lor B)$ & $A$                                             & \index{Absoption}Absorption  \\
      $A \lor (\mathbf{A} \land B)$ & $A$                                             &             \\
      $A \Rightarrow B$                          & $\mathbf{\neg A} \lor B$           & \index{Elimination}Elimination \\
      $\neg (A \Rightarrow B)$                   & $A \land \neg B$                                &             \\
      $A \Leftrightarrow B$                      & $(A \Rightarrow B) \land (B \Rightarrow A)$     &
    \end{tblr}
  }
\end{mzImportant}

\subsection{Axiomatik}

\index{Axiom}
\paragraph{Axiome} als wahr angenommene Aussagen; an Nützlichkeit gemessen.

Anspruch, aber nach \textsc{Gödels} \index{Unvollständigkeitssatz} Unvollständigkeitssatz nicht möglich:

\begin{itemize}
  \item Unabhängig
  \item Vollständig
  \item Widerspruchsfrei
\end{itemize}

\index{Prädikatenlogik}
\section{Prädikatenlogik}

\index{Quantor}
\paragraph{Quantoren} Innerhalb eines Universums:

\begin{description}
  \item [Existenzq. $\exists$]
        \index{Existenzquantor}
        ,,Mind. eines``

  \item [Individuum $\exists!$]
        \index{Individuum}
        ,,Genau eines``

  \item [Allq. $\forall$ ]
        \index{Allquantor}
        ,,Für alle``
\end{description}

\paragraph{Quantitative Aussagen}

\begin{description}
  \item[Erfüllbar]
    \index{Erfüllbar}
    $\exists x F(x)$

  \item[Widerlegbar]
    \index{Widerlegbar}
    $\exists x \neg F(x)$

  \item[Tautologie]
    \index{Tautologie}
    $\top = \forall x F(x)$
    (alle Schlussregeln)

  \item[Kontradiktion]
    \index{Kontradiktion}
    $\bot = \forall x \neg F(x)$
\end{description}

\mzScale{.25}{
  \begin{tikzpicture}[fill=background,fill opacity=0.5,text opacity=1,scale=1/12]
    % Erfüllbar.
    \fill (0,0) circle (12);
    % Tautologie.
    \filldraw (-4,0) circle (4) node {T};
    % Widerlegbar.
    \filldraw (16,0) circle (12) node[above=8] {W};
    % Kontradiktion.
    \filldraw (20,0) circle (4) node {K};
    % Erfüllbar Outline
    \draw (0,0) circle (12) node[above=8] {E};
  \end{tikzpicture}
}

\mzGraphic{
  \begin{tblr}{
    cells = {c},
    column{2} = {r},
    cell{4}{2} = {r=2}{},
    hline{1-2,6} = {-}{},
      }
    \textbf{Klassische Tautologien}           & Bezeichnung              \\
    $A \lor \neg A$                           & \index{Ausgeschlossenes Drittes}Ausgeschlossenes Drittes \\
    $A \land (A \Rightarrow B) \Rightarrow B$ & \index{Modus ponens}Modus ponens             \\
    $(A \land B) \Rightarrow A$               & \index{Abschwächung}Abschwächung             \\
    $A \Rightarrow (A \lor B)$                &
  \end{tblr}
}

\paragraph{Negation} (\textsc{De-Morgan})
\index{Negation}

\begin{align*}
  \neg \exists x F(x) \Leftrightarrow \boldsymbol{\forall} x \boldsymbol{\neg} F(x) \\
  \neg \forall x F(x) \Leftrightarrow \boldsymbol{\exists} x \boldsymbol{\neg} F(x)
\end{align*}

\subsection{Häufige Fehler}

\begin{itemize}
  \item $U = \emptyset^\complement$
        nicht notwendig

  \item $\exists x (P(x) \Rightarrow Q(x)) \not \Rightarrow \exists x P(x)$

  \item $\neg \exists x \exists y P(x, y) \Leftrightarrow \forall x \neg \exists y P(x, y)$
\end{itemize}

\section[Beweistechniken]{Beweistechniken\hfill\emph{\qedsymbol}}

\paragraph{Achtung:} Aus falschen Aussagen können wahre \emph{und} falsche Aussagen folgen.

\begin{description}
  \item[Direkt $A \Rightarrow B$]
    \index{Direkter Beweis}
    Angenommen $A$, zeige $B$.
    Oder: Angenommen $\neg B$, zeige $\neg A$ \linebreak (\emph{Kontraposition}\index{Kontraposition}).

    $$
      (A \Rightarrow B) \Leftrightarrow (\neg B \Rightarrow \neg A)
    $$

  \item[Fallunters.]
    \index{Fallunterscheidung}
    Aufteilen, lösen, zusammenführen.
    O.B.d.A\index{O.B.d.A} = ,,Ohne Beschränkung der Allgemeinheit``

  \item[Widerspruch $(\neg A  \Rightarrow \bot) \Rightarrow A$]
    \index{Widerspruch}
    Angenommen $A \land \mathbf{\neg B}$, zeige Kontradiktion. (Reductio ad absurdum)\index{Reductio ad absurdum}

  \item[Ring] (Transitivität der Implikation)
    \begin{align*}
             & A \Leftrightarrow B \Leftrightarrow C \Leftrightarrow \cdots            \\
      \equiv & A \Rightarrow B \Rightarrow C \Rightarrow \cdots \mathbf{\Rightarrow A}
    \end{align*}

  \item[Induktion $F(n) \quad \forall n \geq n_0 \in \mathbb{N}$]
    \index{Induktion}\

    \begin{enumerate}
      \item
            \begin{description}
              \item[Anfang:]
                \index{Induktionsanfang}
                Zeige $F(n_0)$.
            \end{description}

      \item \begin{description}
              \item [Schritt:]
                    \index{Induktionsschritt}
                    Angenommen
                    $F(n)$ (Hypothese\index{Induktionshypothese}),
                    zeige $F(n + 1)$ (Behauptung\index{Induktionsbehauptung}).
              \item [Starke Induktion:]
                    \index{Starke Induktion}
                    Angenommen
                    $F(k) \quad \forall n_0 \leq k \leq n \in \mathbb{N}$.
            \end{description}
    \end{enumerate}
\end{description}

\subsection{Häufige Fehler}

\begin{itemize}
  \item Nicht voraussetzen, was zu beweisen ist

  \item Äquival. von Implikat. unterscheiden (Zweifelsfall immer Implikat.)
\end{itemize}