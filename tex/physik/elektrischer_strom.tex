\chapter{Elektrischer Strom}


\section{Elektrisches Feld}

\paragraph{Elektrische Ladung}

$$Q = N * e_0 = [C] = [As]$$

\begin{itemize}
  \item $1C = (6,242 * 10^{18}) * e_0$
  \item $e_0 = 1,602 * 10^{-19} C$
\end{itemize}

\paragraph{Culombsches Gesetz}

$$\vec{F} = \frac{1}{4 \pi \epsilon} * \frac{Q_1 * Q_2}{r^2} * (\vec{r_0}) = [N]$$

\begin{itemize}
  \item $\epsilon_0 = 8,854 * 10^{-12} \frac{C^2}{Nm^2}$
  \item Ungleiche Ladungen ($Q$) ziehen sich an, gleich sto\ss en sich ab
  \item $F \propto 1/r^2$
\end{itemize}

\paragraph{Elektrisches Feldstärke}
% TODO: TikZ Bild vom Elektrischen Feld.

$$\vec{E} = \frac{\vec{F}}{q} = \left[\frac{V}{m}\right] = \left[\frac{N}{C}\right]$$

\begin{itemize}
  \item Kraft, die Probeladung $q$ erfährt
  \item Feldlinien von kleineren Ladung zur grö\ss eren Ladung (Positiv zu Negativ); gleich der wirkenden Kraftrichtung
\end{itemize}

\paragraph{Elektrisches Potential}

$$\varphi(r) = \frac{Q}{4 \pi \epsilon r} = (-\int_\infty^r \frac{Q}{4 \pi \epsilon r^2} dr)$$

\begin{itemize}
  \item Punktladung $Q$ erzeugt Potential um sich
  \item Potential ist Steigung des E-Feld $E = - \frac{d\varphi}{dr}$
\end{itemize}

\paragraph{Elektrische Spannung}

\begin{align*}
  U                       & = \frac{W}{q} = [V] = [\frac{Nm}{C}] \\
  U_{r_1 \rightarrow r_2} & = \varphi(r_1) - \varphi(r_2)        \\
\end{align*}

\begin{itemize}
  \item Arbeit um $q$ von $r_1$ nach $r_2$ zu bewegen $W_{r_1 \rightarrow r_2} = \int_{r_1}^{r_2} \vec{F}dr$
\end{itemize}

\subsection{Elektrischer Strom}

$$I = Q/t = [A] = \left[\frac{C}{s}\right]$$

\begin{itemize}
  \item Gleichmä\ss ig gerichteter Fluss von Elektronen von Minus nach Plus (,,physikalisch``)
  \item $1A = \frac{1}{1,602} * 10^{19}$ Elektronen pro Sekunde
  \item $\Rightarrow Q = \int_0^t i(t)dt$
\end{itemize}

\paragraph{Elektrische Arbeit}

$$W = I * t * U = [Ws] = [J]$$

\begin{itemize}
  \item Ladungstransport über Zeit mit Spannung
  \item Am Widerstand freigesetzte Energie $W = \frac{U^2}{R} * t$
\end{itemize}

\paragraph{Elektrische Leistung}

$$P = \frac{W}{t} = U * I = [W] = [VA]$$

\begin{itemize}
  \item Arbeit pro Zeit
  \item Am Widerstand $P = U^2 / R$
\end{itemize}

\subsection{Elektrisches Netz}

% TODO: TikZ Schema.

Strom flie\ss t per Definition (,,technisch``) von Plus ($+$) nach Minus ($-$)

\begin{description}
  \item[Generator $G$] gibt Energie frei $W < 0$
  \item[Verbraucher $R$] verbraucht E. $W > 0$
  \item[Verbindungsleitungen] nach Kirchhoff:
    \paragraph{Knoten $K$} Verzweigung der Verbindungsleitung
    $$\sum_{i \in K} I_i = 0 A$$

    \begin{itemize}
      \item Stromrichtung einmalig willkürlich festlegen
      \item Eingehende Ströme addieren, ausgehende subtrahieren
      \item Ladungen werden nicht angehäuft $\Rightarrow$ Eingehender $=$ ausgehender Strom auch bei Bauteilen
    \end{itemize}

    \paragraph{Masche $M$} Geschlossener Pfad ohne Knotenwiederholung
    $$\sum_{k \in M} U_k = 0 V$$

    \begin{itemize}
      \item Pfad startet im Knoten
      \item Vorher Spannungsrichtung ($=$ Stromrichtung) einzeichnen
      \item Spannungsrichtung in Maschenrichtung addieren, entgegen Maschenrichtung (Quellen) subtrahieren
    \end{itemize}
\end{description}

\subsection{Lösen Linearer Gleichungssysteme}

Kirchhoff'sche Sätze schaffen Lineares Gleichungssystem der Form

$$A x = b$$

\begin{itemize}
  \item $x$ ist der gesuchte Vektor der Ströme $I_k = x_k$
  \item $A$ ist die Matrix der Koeffizienten (Widerstände)
  \item $b$ sind vom Strom unabhängige Grö\ss en (Spannungen, $0A$ im Knoten)
\end{itemize}

\paragraph{Matrixmul.}
\index{Matrixmultiplikation}
$(m \times n)(n \times p) = (m \times p)$

\begin{mzImportant}
  $$(AB)_{ij} = \sum_{k=1}^m a_{ik}b_{kj}$$

  (Zeile $\times$ Spalte)
\end{mzImportant}

\paragraph{Determinante}

\begin{align*}
  \det A & = \sum_{i=1}^n (-1)^{i + j} * a_{ij} * \det A_{ij} \\
         & = \sum_{j=1}^n (-1)^{i + j} * a_{ij} * \det A_{ij} \\
\end{align*}

\begin{itemize}
  \item Für Matrix $A \in \mathbb{R}^n$
  \item ,,Entwickeln`` nach $i$-ter Zeile oder $j$-ter Spalte
  \item $A_{ij} =$ Matrix $A$ ohne $i$-te Zeile und $j$-te Spalte
  \item Zeile/Spalte wählen mit viel $a_{ij} = 0$, damit $\det A_{ij}$ nicht berechnet werden muss
\end{itemize}

\paragraph{$(2 \times 2)$ Matrix}

$$\det A = \begin{vmatrix}
    a & c \\
    b & d
  \end{vmatrix} = ad - bc$$

\paragraph{$(3 \times 3)$ Matrix (Regel von Sarrus)}

\mzScale{.7}{
  \begin{tikzpicture}
    \matrix (sarrus) [
      matrix of math nodes,
      ampersand replacement=\&,
      column sep=1em,
      row sep=1em
    ] {
      a_{11} \& a_{12} \& a_{13} \& a_{11} \& a_{12} \\
      a_{21} \& a_{22} \& a_{23} \& a_{21} \& a_{22} \\
      a_{31} \& a_{32} \& a_{33} \& a_{31} \& a_{32} \\
    };

    \path
    (sarrus-1-3.north east) edge[dashed] (sarrus-3-3.south east)
    (sarrus-3-1) edge[color=primary] (sarrus-2-2)
    (sarrus-2-2) edge[color=primary] (sarrus-1-3)
    (sarrus-3-2) edge[color=primary] (sarrus-2-3)
    (sarrus-2-3) edge[color=primary] (sarrus-1-4)
    (sarrus-3-3) edge[color=primary] (sarrus-2-4)
    (sarrus-2-4) edge[color=primary] (sarrus-1-5)
    (sarrus-1-1) edge         (sarrus-2-2)
    (sarrus-2-2) edge         (sarrus-3-3)
    (sarrus-1-2) edge         (sarrus-2-3)
    (sarrus-2-3) edge         (sarrus-3-4)
    (sarrus-1-3) edge         (sarrus-2-4)
    (sarrus-2-4) edge         (sarrus-3-5);

    \foreach \c in {1,2,3} {\node[anchor=south] at (sarrus-1-\c.north) {$+$};};
    \foreach \c in {1,2,3} {\node[anchor=north] at (sarrus-3-\c.south) {$-$};};
  \end{tikzpicture}
}

\paragraph{Cramer'sche Regel}

\begin{align*}
  x_k & = (I_k) = \frac{\det A_k}{\det A} \quad \det A_k \neq 0        \\
  A_k & = (a_1 \mid \dots \mid a_{k-1} \mid b \mid a_{k + 1} \mid a_m)
\end{align*}

\begin{itemize}
  \item $A_k$ ist Matrix $A$ mit Vektor $b$ statt $k$-ter Spalte
  \item Lösbar $\Leftrightarrow \det A \neq 0$
\end{itemize}

\section{Elektromagnetisches Feld}
% TODO: TikZ Grafik.

Stromdurchflossene Leiter erzeugen Magnetfelder orthogonal zur Flussrichtung:

\paragraph{Rechte-Hand-Regel}

\begin{itemize}
  \item Daumen in (technische) Stromrichtung (Vektorprodukt)
  \item Gekrümmte Finger in Magnetfeldrichtung (Norden)
  \item Zeigefinger in Magnetfeldrichtung $\Rightarrow$ Mittelfinger in Kraftwirkung auf Leiter
\end{itemize}

\paragraph{Magnetische Feldstärke}

$$\vec{H} = \frac{\vec{\Theta}}{s} = \frac{I}{s} * \vec{e_s} = \frac{I}{2\pi r} * \vec{e_s} = \left[\frac{A}{m}\right]$$

\begin{itemize}
  \item Erzeugt durch stromdurchflossene Leiter $I$
  \item $\vec{e_s}$ Einheitsvektor tangential zum Umfang
\end{itemize}

\paragraph{1. \textsc{Maxwell}'sche Gleichung: Durchflutungsgesetz}

$$\oint \vec{H} ds = \iint_A \vec{j} dA$$

Geschlossene magnetische Feldlinien werden von Strom durchflutet

\paragraph{Magnetische Spannung}

$$\vec{\Theta}_{s_1 \rightarrow s_2} = \int_{s_1}^{s_2} \vec{H} ds = \vec{I} = [A]$$

\begin{itemize}
  \item Zwischen Umfang $s_1$ (z.B $=2\pi r_1$) und $s_2$
\end{itemize}

\paragraph{Magnetische Flussdichte}

$$B = \mu_0 * \mu_r * \vec{H} = [T] = \left[\frac{Vs}{m^2}\right]$$

\begin{itemize}
  \item $\mu_0 = 1,2566 * 10^{-6} \frac{Vs}{Am}$
\end{itemize}

\paragraph{Relative Permeabilität: Hysteresekurve}

\begin{itemize}
  \item Feromagnetische Stoffe $\mu_r = 10^2 \dots 10^5$ oder nicht konstant
  \item Speichern magnetische Zustände
\end{itemize}
% TODO: TikZ Grafik der Hysteresekurve.

\begin{description}
  \item[Remanenzpunkt $B_r$] Magnetische Flussdichte $B_r$, die \emph{nach} ($H = 0$) einer Magnetisierung besteht
  \item[Koerzitivfeldstärke $-H_c$] Feldstärke um Material zu entmagnetisieren
\end{description}

\paragraph{Wechselschriftverfahren}

\begin{itemize}
  \item $\mathbf{1}$ Permanenter Richtungswechsel des Stroms (durch antiparalleles Magnetfeld zum vorherigen Takt)
  \item $\mathbf{0}$ keine Veränderung des Stroms
\end{itemize}

\begin{description}
  \item[Lesen] Bewegung des magnetisierten Mediums induziert Strom bei antiparalleln Magnetfeld zum vorherigen Takt (Veränderung), bleibt $0$ bei keiner Veränderung
  \item[Schreiben] Positiver und negativer Strom magnetisiert Medium antiparallel
\end{description}

\paragraph{Kraftwirkung des magnetischen Feldes}

$$\vec{F} = \mu * l * \vec{I} \times \vec{H} = l * \vec{I} \times \vec{B}$$

\begin{itemize}
  \item Kinetische Kraft auf stromdurchflossene Leiter $\vec{I}$ der Länge $l$
  \item $|F| = \mu * l * I * H = l * I * B$
\end{itemize}

\paragraph{Kreuzprodukt} $\vec{a} \times \vec{b}$

$$\begin{pmatrix}
    a_1 \\ a_2 \\ a_3
  \end{pmatrix} \times \begin{pmatrix}
    b_1 \\ b_2 \\ b_3
  \end{pmatrix} = \begin{pmatrix}
    a_2b_3 - a_3b_2 \\
    a_3b_1 - a_1b_3 \\
    a_1b_2 - a_2b_1 \\
  \end{pmatrix}$$

\paragraph{Elektromagnetische Induktion}

$$U_i = - \frac{d \iint \vec{B} d\vec{A}}{dt} = - \frac{d \Phi}{dt}$$

\begin{itemize}
  \item Umgekehrt induziert Bewegung eines Leiters im Magnetfeld eine Spannung
\end{itemize}

\paragraph{Magnetischer Fluss}

$$\Phi = \iint \vec{B} d \vec{A} = [Wb] = [T * m^2]$$

\begin{itemize}
  \item Homogenes Magnetfeld $\Phi = \vec{B} * \vec{A}$
  \item Leiter im Winkel zum geradlinigen Magnetfeld $\Phi = B * A * \cos \varphi$
\end{itemize}

\section{Wechselstrom}

Die Rotation eines Leiters in einem Magnetfeld induziert eine Wechselspannung und einen Wechselstrom:

\begin{gather*}
  u(t) = \hat{u} * \sin(\omega t) \\
  i(t) = \hat{i} * \sin(\omega t)
\end{gather*}

\begin{itemize}
  \item Frequenz $f = 1/T$ (Anzahl der Perioden pro Zeiteinheit)
  \item Drehgeschwindigkeit $\omega = \frac{\varphi}{t} = 2 \pi f$ (Anzahl der Perioden auf $2\pi$ Weg)
\end{itemize}

\paragraph{Kenngrö\ss en}

\begin{description}
  \item[Linearer Mittelwert] (Durchschnitt)
    $$\overline{Y} = \frac{\int y(x)dx}{\int dx} \quad \overline{I} = \frac{1}{T} \int^T i(t) dt$$

    \begin{itemize}
      \item Gemä\ss~Normung $= 0A$
    \end{itemize}

  \item[Gleichrichtwert] (Durchschnitt des Betrag)
    $$|\overline{I}| = \frac{1}{T} \int^T |i(t)| dt$$

  \item[Effektivwert] (Leistung Gleichstrom)
    $$I_\text{eff.} = \sqrt{\frac{1}{T} \int^T i^2(t) dt} $$

    \begin{itemize}
      \item Sinusförmig: $I_\text{eff.} = \frac{\hat{i}}{\sqrt{2}}$, $U_\text{eff.} = \frac{\hat{u}}{\sqrt{2}}$
    \end{itemize}

  \item[Formfaktor] $k = \frac{I_\text{eff.}}{|\overline{I}|}$

    \begin{itemize}
      \item Sinusförmig: $k = \frac{\pi}{\sqrt{8}} \approx 1,1107$
      \item Rechteck: $k = 1$
    \end{itemize}
\end{description}

\subsection{Komplexe Wechselstromrechnung}

$$\underline{\hat{u}} = \hat{u} * (\cos \hat{\varphi} + j \sin \hat{\varphi}) = \hat{u} * e^{j\hat{\varphi}}$$

\begin{itemize}
  \item Komplexe Amplitude mit Phasensprung $\hat{\varphi}$
\end{itemize}

\paragraph{Komplexe Zahlen}

$$\underline{c} = x + jy = re^{j \varphi}$$

\begin{itemize}
  \item $r = \sqrt{x^2 + y^2}$
  \item $\varphi = \arctan \frac{y}{x}$
  \item $x = r \cos \varphi$, $y = r \sin \varphi$
\end{itemize}

\begin{description}
  \item[Addition]
    \begin{align*}
      \underline{U}_1 + \underline{U}_2 = & \Re(\underline{U}_1) + \Re(\underline{U}_2)    \\
      +                                   & j(\Im(\underline{U}_1) + \Im(\underline{U}_2))
    \end{align*}
  \item[Multiplikation]
    $$\underline{U}_1 * \underline{U}_2 = r_1 r_2 e^{j(\varphi_1 + \varphi_2)}$$
    % \item[Division]
    %   $$\frac{\underline{U}_1}{\underline{U}_2} = \frac{U_1}{U_2} e^{j(\varphi_1 - \varphi_2)}$$
\end{description}

\paragraph{Impedanz}

$$\underline{Z} = \underline{U} / \underline{I} = R + jX = [\Omega]$$

\begin{description}
  \item[Scheinwiderstand] $|\underline{Z}|$
  \item[Wirkwiderstand] $R$ (Wirkleistung gleich Gesamtleistung bei Ohmschen Widerständen)
  \item[Blindwiderstand] $X$ (Blindleistung bei Induktiven und Kapazitiven Bauteilen zum Aufbau des Feldes)
\end{description}

\section{Signale}

\paragraph{Bandbreite} Grö\ss e des Frequenzbereichs in dem ohne wesentliche Störeffekte übertragen werden kann

$$W = [Hz]$$

\paragraph{Wei\ss es Rauschen} Signal aller zufälligen Störeffekte

\paragraph{Rauschabstand} Verhältnis Signalstärke zu Rauschstärke (bzw. Leistung)

$$10 \log_{10} \frac{S}{N} = [dB]$$

\paragraph{Kanalkapazität} Maximale Informationsmenge die auf einem Kanal übertragen wird (Satz von \textsc{Shannon})

$$C = W * \log_2 (1 + \frac{S}{N}) = [Bit/s]$$

\paragraph{Signalarten}

\begin{description}
  \item[Daten]\ \begin{itemize}
    \item analog (zeitkontin., wertekontin)
    \item digital (zeitdiskret., wertdiskret.)
  \end{itemize}

  \item[Signale] (meist zeitkontin) \begin{itemize}
      \item analog (wertekontin.)
      \item digital (wertediskret.)
    \end{itemize}
\end{description}

\mzGraphic{
  \begin{tblr}{
      row{2} = {c},
      cell{1}{1} = {c=2,r=2}{},
      cell{1}{3} = {c=4}{c},
      cell{3}{1} = {r=4}{c},
      cell{3}{3} = {c},
      cell{3}{4} = {c},
      cell{3}{5} = {c},
      cell{3}{6} = {c},
      cell{4}{3} = {c},
      cell{4}{4} = {c},
      cell{4}{5} = {c},
      cell{4}{6} = {c},
      cell{5}{3} = {c},
      cell{5}{4} = {c},
      cell{5}{5} = {c},
      cell{5}{6} = {c},
      cell{6}{3} = {c},
      cell{6}{4} = {c},
      cell{6}{5} = {c},
      cell{6}{6} = {c},
      vline{2-3} = {3-6}{},
      vline{3} = {2-6}{},
      hline{2} = {3-6}{},
      hline{3} = {2-6}{},
    }
    &  & \textbf{Ergebnissignal} &  &  & \\
    &  & {\itshape zeitkontin.\\wertkontin.} & {\itshape zeitdiskret.\\wertkontin.} & {\itshape zeitkontin.\\wertdiskret.} & {\itshape zeitdiskret.\\wertdiskret.}\\
    \begin{sideways}\textbf{Eingangssignal}\end{sideways} & {\itshape zeitkontin.\\wertkontin.} &  & Abtastung & Quantisierung & A/D-Wandlung\\
    & {\itshape zeitdiskret.\\wertkontin.} & Interpolation &  &  & Quantisierung\\
    & {\itshape zeitkontin.\\wertdiskret.} & Glättung &  &  & Abtastung\\
    & {\itshape zeitdiskret.\\wertdiskret.} & D/A-Wandlung &  & Interpolation &
  \end{tblr}
}

% TODO: A/D-Wandler + Operationsverstärker.

\paragraph{Auffrischen von Signalen} in Abständen

\begin{description}
  \item[Verstärker] (analog)
  \item[Regenerator] (digital)
\end{description}

\paragraph{Satz von \textsc{Nyquist}} Ein Signal der Frequenz $f$ kann mit einer Abtastfrequenz $\geq 2f$ rekonstruiert werden

\subsection{\textsc{Fourier}-Analyse}

Konvertierung des Zeitraums mit Fourierreihen $a_n$, $b_n$ in Frequenzraum über $n$-Harmonische mit Harmonischer Analyse

\begin{align*}
  F(t) = \frac{a_0}{2} + & \sum_{n = 1}^{\infty} a_n \cos (2\pi nft)             \\
  +                      & \sum_{n = 1}^{\infty} b_n \sin (2\pi nft)             \\
  = \frac{a_0}{2} +      & \sum_{n = 1}^{\infty} c_n \cos (2\pi nft - \varphi_n)
\end{align*}

\begin{itemize}
  \item Grundfrequenz $f = 1/T$
  \item $a_n = \frac{2}{T} \int_{0}^{T} F(t) \cos (2\pi nft) dt$
  \item $b_n = \frac{2}{T} \int_{0}^{T} F(t) \sin (2\pi nft) dt$
  \item $c_n = \sqrt{a_n^2 + b_n^2}$, $\varphi_n = \arctan \frac{b_n}{a_n}$
  \item $f$ muss überall monoton und stetig bzw. existieren an jeder Unstetigkeitsstelle die links- und rechtsseitigen Grenzwerte (\textsc{Dirichletsche}-Bedingung)
\end{itemize}

\section{Häufige Fehler}

\begin{itemize}
  \item Vektor und Skalare Formeln mischen
  \item Gesamtkapazitäten statt Kirchhoff
  \item $mm^3 = (10^{-3} m)^3 = 10^{-9} m^3$
  \item $1/k\Omega = m\Omega$
\end{itemize}