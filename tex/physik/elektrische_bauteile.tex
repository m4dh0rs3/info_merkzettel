\chapter{Elektrische Bauteile}

\section{Elektrischer Leiter}

\paragraph{Elektrische Flussdichte}

$$D = \frac{Q}{A} = \left[\frac{C}{m^2}\right]$$

\begin{itemize}
  \item Frei bewegliche Ladungsträger verteilen sich gleichmä\ss ig auf der Oberfläche
  \item $\Rightarrow Q = A * \iint_A Dd$
  \item $\vec{D} = \epsilon_0 * \epsilon_r * \vec{E}$ ($r$ raumfüllendes Material)
\end{itemize}

\paragraph{Metallischer Leiter}

$$R = \rho \frac{l}{A}$$

\begin{itemize}
  \item Linearer Widerstand, abhängig vom Material $\rho$
  \item $\rho = [\Omega \frac{mm^2}{m}] \propto$ Länge, kleinere Oberfläche
\end{itemize}

\section{Ohmsch: Lineare Widerstände}

$$U = R * I$$

\begin{itemize}
  \item Kurz ,,Uri``
  \item Strom $\propto$ Spannung, kleinerer Widerstand
\end{itemize}

$$R = [\Omega] = \left[\frac{V}{A}\right]$$

\paragraph{Leitwert} $G = 1/R = [S] = \left[\frac{A}{V}\right]$

\paragraph{Schaltung}

\begin{description}
  \item[Reihe] $R_G = \sum R_k$
    \begin{itemize}
      \item $I_k = I \Rightarrow U_k = I * R_k$
    \end{itemize}

  \item[Parallel] $R_G = 1 / \sum \frac{1}{R_k}$
    \begin{itemize}
      \item $U_k = U \Rightarrow I_k = U/R_k$
    \end{itemize}
\end{description}

\paragraph{Kennlinie} Graph $I(U)$
% TODO: TikZ Graph der Kennlinie

\begin{itemize}
  \item Je flacher desto stärker der Widerstand
  \item Für nicht-lineare Graphen gilt das Ohmsche Gesetz nicht!
\end{itemize}

\section{Kapazitiv: Kondensator}

% TODO: TikZ Grafik des Kondensators mit metallischen Hohlraum.

$$Q = C * U$$

(,,Kuh gleich Kuh``)

$$E = \frac{U}{d} = \frac{D}{\epsilon}$$

\paragraph{Kapazität}

$$C = \frac{\epsilon * A}{d} = [F] = \left[\frac{C}{V}\right]$$

\begin{itemize}
  \item Kondensator speichert elektrische Ladung
  \item $\propto$ Gro\ss e Oberfläche, gro\ss e Permittivität
\end{itemize}

\paragraph{Schaltung}

\begin{description}
  \item[Reihe] $C_G = 1/{\sum \frac{1}{C_k}}$
  \item[Parallel] $C_G = \sum C_k$
\end{description}

\paragraph{Influenz: Faraday'scher Käfig}
Das Innere eines metallischen Hohlraums ist feldfrei.

% TODO: Arten von Kondensatoren.

\section{Induktiv: Spule}

\section{Quellen}

\section{Messgeräte}

\subsection{Voltmeter}

\begin{itemize}
  \item Schaltung in Parallel
\end{itemize}

\subsection{Amperemeter}

\begin{itemize}
  \item Schaltung in Reihe
\end{itemize}

\section{Spezielle Kombinationen}

\subsection{Spannungsteiler}

$$U_A = \frac{U_0}{\frac{R_1}{R_2} + 1}$$

\paragraph{Potentiometer} $R_1 = R - R_2$

$$\Rightarrow U_A = U_0 * \frac{R_2}{R}$$

\paragraph{Potentiometer unter Last $R_L$}
$R_1 = R - (R_2 \parallel R_L)$

$$\Rightarrow U_A = U_0 * \frac{R_2}{R} * \frac{R_L}{R_L + R_2}$$

\subsection{Transformator}

\subsection{Schwingkreis}