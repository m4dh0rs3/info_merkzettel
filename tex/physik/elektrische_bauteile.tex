\chapter{Elektrische Bauteile}

\section{Elektrischer Leiter}

\paragraph{Elektrische Flussdichte}

$$D = \frac{Q}{A} = \left[\frac{C}{m^2}\right]$$

\begin{itemize}
  \item Frei bewegliche Ladungsträger verteilen sich gleichmä\ss ig auf der Oberfläche
  \item $\Rightarrow Q = A * \iint_A Dd$
  \item $\vec{D} = \epsilon_0 * \epsilon_r * \vec{E}$ ($r$ raumfüllendes Material)
\end{itemize}

\paragraph{Elektrische Stromdichte}

$$J = \frac{I}{A}$$

\begin{itemize}
  \item Querschnitt $A$ senkrecht zum Stromfluss $\vec{I}$
  \item $\propto$ Erwärmung des Leiters
  \item Aber: Dünne Leitungen kühlen besser (Verhältnis Querschnitt zu Umfang) $\Rightarrow$ Dicke Leitungen haben geringeres zulässiges $J$
\end{itemize}

\paragraph{Metallischer Leiter}

$$R = \rho \frac{l}{A}$$

\begin{itemize}
  \item Linearer Widerstand, abhängig vom Material $\rho$
  \item $\rho = [\Omega \frac{mm^2}{m}] \propto$ Länge, kleinere Oberfläche
\end{itemize}

\paragraph{Magnetische Feldstärke}

$$\vec{H} = \frac{\vec{I}}{2\pi r} = \left[\frac{A}{m}\right]$$

\section{Ohmsch: Lineare Widerstände}

$$U = R * I$$

\begin{itemize}
  \item Kurz ,,URI``
  \item Strom $\propto$ Spannung, kleinerer Widerstand
\end{itemize}

$$R = [\Omega] = \left[\frac{V}{A}\right]$$

\paragraph{Leitwert} $G = 1/R = [S] = \left[\frac{A}{V}\right]$

\paragraph{Schaltung}

\begin{description}
  \item[Reihe] $R_G = \sum R_k$
    \begin{itemize}
      \item $I_k = I \Rightarrow U_k = I * R_k$
    \end{itemize}

  \item[Parallel] $R_G = 1 / \sum \frac{1}{R_k}$
    \begin{itemize}
      \item $U_k = U \Rightarrow I_k = U/R_k$
    \end{itemize}
\end{description}

\paragraph{Kennlinie} Graph $I(U_A)$
% TODO: TikZ Graph der Kennlinie

\begin{itemize}
  \item Je flacher desto stärker der Widerstand
  \item Für lineare Bauteile: Nullstelle $I(U_A) = 0A$ und Schnittpunkt mit der $I$-Achse bestimmen $I(0V) = I_0$
  \item Für nicht-lineare Graphen $R(U, I) = U/I$ gilt das Ohmsche Gesetz nicht!
\end{itemize}

\paragraph{Arbeitspunkt} Schnittpunkt der Kennlinien $I_1 (U_A) = I_2 (U_A)$

\begin{itemize}
  \item Bestimmung der dynamischen Austarierung nicht-linearer Bauteile
  \item Kennlinie in Abhängigkeit der Spannung am Bauteil, nicht der Quellspannung!
\end{itemize}

\paragraph{Energierverbrauch}

$$W_R = t * RI^2$$

\subsection{Wechselstrom}

$$i(t) = \frac{\hat{u}}{R} * \sin \omega t$$

\begin{description}
  \item[Maximalstrom] $\hat{i} = \frac{\hat{u}}{R}$
  \item[Widerstand] $R = \frac{\hat{u}}{\hat{i}}$
\end{description}

\paragraph{Komplexwertig}

\begin{gather*}
  \underbar{u}(t) = R * \underline{i}(t) \\
  \underline{U}_\text{eff.} = R * \underline{I}_\text{eff.}
\end{gather*}

\paragraph{Impedanz}

$$\underline{Z} = R + 0j$$

\section{Kapazitiv: Kondensator}

% TODO: TikZ Grafik des Kondensators mit metallischen Hohlraum.

$$Q = C * U$$

(,,Kuh gleich Kuh``)

$$E = \frac{U}{d} = \frac{D}{\epsilon}$$

\paragraph{Kapazität}

$$C = \frac{\epsilon * A}{d} = [F] = \left[\frac{C}{V}\right]$$

\begin{itemize}
  \item Kondensator speichert elektrische Ladung
  \item $\propto$ Gro\ss e Oberfläche, gro\ss e Permittivität, kleiner Abstand
  \item Durchschlagfestigkeit $E_d = U_d / d$
\end{itemize}

\paragraph{Energie im Elektrischen Feld}

$$W = \frac{1}{2} C * U^2$$

\paragraph{Influenz: Faraday'scher Käfig}
Das Innere eines metallischen Hohlraums ist feldfrei.

\paragraph{Schaltung}

\begin{description}
  \item[Reihe] $C_G = 1/{\sum \frac{1}{C_k}}$
  \item[Parallel] $C_G = \sum C_k$
\end{description}
% TODO: Arten von Kondensatoren.

\paragraph{Ladevorgänge}

\begin{description}
  \item[Einschalten]\
  \begin{itemize}
    \item $U_C = U * (1 - e^{- \frac{t}{R * C}})$
    \item $I_C = \frac{U}{R} * e^{- \frac{t}{R * C}}$
  \end{itemize}

  \item[Ausschalten]\
  \begin{itemize}
    \item $U_C = U * e^{- \frac{t}{R * C}}$
    \item $I_C = \frac{U}{R} * e^{- \frac{t}{R * C}}$
  \end{itemize}
\end{description}

\subsection{Wechselstrom}

$$i(t) = C \hat{u} * \omega \cos (\omega t)$$

\begin{description}
  \item[Maximalstrom] $\hat{i} = C \hat{u} * \omega$
  \item[Phasensprung] von $\pi/2$
  \item[Widerstand] $R_C = \frac{1}{\omega * C}$
\end{description}

\paragraph{Komplexwertig}

\begin{gather*}
  \underline{i}(t) = C \underline{u}(t) * j\omega \\
  \underline{u}(t) = \underline{i}(t) / (C * j\omega) \\
  \underline{I}_\text{eff.} = C \underline{U}_\text{eff.} * j\omega
\end{gather*}

\paragraph{Impedanz}

$$\underline{Z} = 0 -j \frac{1}{\omega C}$$

\section{Induktiv: Spule}

Die durch die Spannungsveränderung (z.B Anlegung) induzierte Spannung wirkt der Spannung entgegen (Lenzsche Regel):

$$U = L * \frac{dI}{dt}$$

Ein magnetischer Fluss induziert in der Spule eine Spannung:

$$\Phi = L * I$$

\paragraph{Selbstinduktivität}

$$L = [H] = \left[\frac{Vs}{A}\right]$$

\begin{itemize}
  \item $1H$ wenn bei einer gleichförmigen Stromveränderung von $1A$ in $1s$ eine Selbstinduktion von $1V$ erzeugt wird
  \item $\propto N^2$ Quadrat der Windungszahl
\end{itemize}

\paragraph{Magnetische Feldstärke}

$$H = \frac{\Theta}{l} = \frac{I * N}{\sqrt{l^2 + D^2}}$$

\begin{itemize}
  \item Feldstärke im Zentrum eines Zylinder des Durchmessers $D$
\end{itemize}

Angenommen $l \gg D$ ,,schlank``

$$H = \frac{I * N}{l}$$

\paragraph{Energie im Magnetfeld}

$$W = \frac{1}{2} L * I^2$$

\paragraph{Ladevorgänge}

\begin{description}
  \item[Einschalten] $I_L = \frac{U}{R} * (1 - e^{-t * \frac{R}{L}})$
  \item[Ausschalten] $I_L = \frac{U}{R} * e^{-t * \frac{R}{L}}$
\end{description}

\subsection{Wechselstrom}

$$i(t) = \frac{\hat{u}}{\omega * L} * \sin (\omega t - \frac{\pi}{2})$$

\begin{description}
  \item[Maximalstrom] $\hat{i} = \frac{\hat{u}}{\omega * L}$
  \item[Phasensprung] von $-\pi/2$
  \item[Widerstand] $R_L = \omega * L$
\end{description}

\paragraph{Komplexwertig}

\begin{gather*}
  \underline{u}(t) = L * \underline{i}(t) * j\omega \\
  \underline{U}_\text{eff.} = L * \underline{I}_\text{eff.} * j\omega
\end{gather*}

\paragraph{Impedanz}

$$\underline{Z} = 0 + j \omega L$$

\section{Quellen}

\subsection{Spannungsquelle}

Feste Spannung $U_Q$

\begin{itemize}
  \item Ideal: $\lim_{R_L \rightarrow 0} I \geq \infty$
\end{itemize}

\paragraph{Klemmspannung} Tatsächliche Spannung mit geringem Innenwiderstand $R_{iQ}$ in Reihe

$$U = U_Q - I * R_{iQ} \Rightarrow I = \frac{U_Q}{R_{iQ} + R_L}$$

\begin{description}
  \item[Leerlauf] Nicht geschlossen, $I = 0$
  \item[Kurzschluss] Ohne Last geschlossen; da $R_{iQ}$ gering $\Rightarrow$ gefährlich hohe Leistung $P = U_Q^2 / R_{iQ}$
\end{description}

\subsection{Stromquelle}

Fester Strom $\forall R_L: I_L = \text{konst.}$

\paragraph{Reale Stromquelle} Hoher Innenwiderstand $R_{iQ}$

\begin{itemize}
  \item $I_L = I_Q - I_{iR}$
  \item Ideal: $\lim_{R_{iQ}} \rightarrow \infty I_L = I_Q$
\end{itemize}

\begin{description}
  \item[Leerlauf] Nicht geschlossen, $U = R_{iQ} * I_Q$
  \item[Kurzschluss] Ohne Last geschlossen; $I_L = I_Q$, $U = 0$
\end{description}

\section{Messgeräte}

\subsection{Spannung: Voltmeter}

\begin{itemize}
  \item Schaltung in Parallel, ohne Amperemeter messen!
  \item Hoher Innenwiderstand $R_{iV} \Rightarrow$ Strom teilt sich auf, Spannung geringer gemessen
  \item $R_{iV} \gg R_L \Rightarrow U_L \approx R_L * I$
\end{itemize}

\subsection{Strom: Amperemeter}

\begin{itemize}
  \item Schaltung in Reihe, ohne Voltmeter messen!
  \item Geringer Innenwiderstand $R_{iA} \Rightarrow$ Strom geringer gemessen
  \item $R_{iA} \ll R_L \Rightarrow I_L \approx U/R_L$
\end{itemize}

\subsection{Widerstand: Fehlerschaltungen}

Zum Messen des Widerstands $R$ wird $I_R$ und $U_R$ benötigt:

\paragraph{Kleiner Widerstand: Stromfehlerschaltung}

\begin{itemize}
  \item Erst Amperemeter in Reihe, dann Widerstand und parallel das Voltmeter
  \item $I \approx I_R$
\end{itemize}

\paragraph{Gro\ss er Widerstand: Spannungsfehlerschaltung}

\begin{itemize}
  \item Erst Voltmeter, dazu parallel der Widerstand und dazwischen in Reihe des Amperemeter
  \item $U \approx U_R$
\end{itemize}

\section{Spezielle Kombinationen}

\subsection{Spannungsteiler}

Die Arbeitsspannung verhält sich zur Quellspannung wie der zweiter Widerstand zum Gesamtwiderstand:

$$\frac{U_A}{U_0} = \frac{R_2}{R_1 + R_2}$$

\begin{itemize}
  \item Setzt Restenergie in Wärme frei
\end{itemize}

\paragraph{Potentiometer} $R_1 = R - R_2$

$$\Rightarrow U_A = U_0 * \frac{R_2}{R}$$

\paragraph{Potentiometer unter Last $R_L$}
$R_1 = R - (R_2 \parallel R_L)$

$$\Rightarrow U_A = U_0 * \frac{R_2}{R} * \frac{R_L}{R_L + R_2}$$

\subsection{Transformator}

$$\frac{U_1}{U_2} = \frac{N_1}{N_2}$$

\begin{itemize}
  \item Wechselspannung der Primärspule induziert Wechselspannung in Sekundärspule
  \item Ideal: Verlustfreier Spannungsteiler, da Energie im Magnetfeld durch Abbau wiedererlangt wird
\end{itemize}

\subsection{Schwingkreis}

$$\omega = \sqrt{\frac{1}{L * C}} = \frac{2\pi}{T}$$

\begin{itemize}
  \item $u_C(t) + u_L(t) = 0V$
  \item (Gedämpft durch Widerstand)
\end{itemize}

\section{Häufige Fehler}

\begin{itemize}
  \item Parallelschaltung von Kondensatoren verhält sich wie Reihenschaltung von Widerständen
\end{itemize}