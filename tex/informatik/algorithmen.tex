\chapter{Algorithmen auf Datenstrukturen}

\index{Algorithmus}
\paragraph{Algorithmus} Handlungsvorschrift aus endlich vielen Einzelschritten zur Problemlösung.

% TODO: Link subsections
\begin{itemize}
  \item Korrektheit (Test-based dev.)
  \item Terminierung (\textsc{Touring})
  \item Effizienz (Komplexität)
\end{itemize}

\index{High-level Sprache}
\index{Low-level Sprache}
\paragraph{Formen (High to low)}
Menschl. Sprache, Pseudocode, Mathematische Ausdrücke, Quellcode, Binärcode

\paragraph{Divide \& Conquer}
\index{Divide \& Conquer}

\begin{description}
  \item [Divide] Zerlegen in kleinere Teilprobleme

  \item [Conquer] Lösen der Teilprobleme mit gleicher Methode (rekursiv)

  \item [Merge] Zusammenführen der Teillösungen
\end{description}

\section{Effizienz}
\index{Effizienz}

Raum/Zeit-Tradeoff: Zwischenspeichern vs. Neuberechnen
\index{Raum/Zeit-Tradeoff}

\mzGraphic{
  \begin{tblr}{
    column{2} = {r},
    hline{1-2,4} = {-}{},
      }
    \textbf{Programmlaufzeit/-allokationen} & \textbf{Komplexität}    \\
    Einfluss äu\ss erer Faktoren      & Unabh.                  \\
    Konkrete Grö\ss e                 & Asymptotische Schätzung
  \end{tblr}
}

\paragraph{Inputgrö\ss e $\mathbf{n}$} Jeweils

\begin{itemize}
  \item Best-case $C_B$
        \index{Best-case Komplexität}

  \item Average-case
        \index{Average-case Komplexität}

  \item Worst-case $C_W$
        \index{Worst-case Komplexität}
\end{itemize}

\subsection{Asymptotische Zeit-/Speicherkomplexität}

\begin{mzImportant}
  \index{Gro\ss-O-Notation}
  \paragraph{Gro\ss-O-Notation}
  Kosten $C_f(n)$ mit $g: \mathbb{N} \rightarrow \mathbb{R} \exists c > 0 \exists n_0 > 0 \forall n \geq n_0$

  \begin{description}
    \item [Untere Schranke] $\boldsymbol{\Omega} (f)$ \\
          \index{Untere Schranke Komplexität}
          $C_f(n) \boldsymbol{\geq} c * g(n)$

    \item [Obere Schranke] $\boldsymbol{O}(f)$ \\
          \index{Obere Schranke Komplexität}
          $C_f(n) \boldsymbol{\leq} c * g(n)$

    \item [Exakte Schranke] $\boldsymbol{\Theta} (f)$ \\
          \index{Exakte Schranke Komplexität}
          $C_f(n) \in \Omega (f) \cap O(f)$ \\
          Polynom $k$ten Grades $\in \Theta (n^k)$
  \end{description}

  (Beweis: $g$ und $c$ finden)
\end{mzImportant}

\mzGraphic{
  \begin{tblr}{
    column{4} = {r},
    cell{1}{3} = {c=2}{c},
    cell{2}{4} = {r=6}{},
    cell{6}{3} = {r=2}{},
    cell{8}{4} = {r=3}{},
    vline{3} = {1,7}{},
    vline{3-4} = {2-10}{},
    hline{1-2,11} = {-}{},
    hline{6,9} = {3}{},
        hline{8} = {3-4}{},
      }
    \textbf{Gro\ss-O}\index{Komplexitätsklassen} & \textbf{Wachstum} & \textbf{Klasse}            &        \\
    $O(1)$          & Konstant          &                            & \begin{sideways}lösbar\end{sideways} \\
    $O(\log n)$     & Logarithmisch     &                            &        \\
    $O(n)$          & Linear            &                            &        \\
    $O(n \log n)$   & Nlogn             &                            &        \\
    $O(n^2)$        & Quadratisch       & Polynomiell $O(n^k)$       &        \\
    $O(n^3)$        & Kubisch           &                            &        \\
    $O(2^n)$        & Exponentiell      & Exponentiell $O(\alpha^n)$ & \begin{sideways}hart\end{sideways} \\
    $O(n!)$         & Fakultät          &                            &        \\
    $O(n^n)$        &                   &                            &
  \end{tblr}
}

\paragraph{Rechenregeln}

\begin{mzImportant}
  \begin{description}
    \item [Elementare Operationen, Kontrollstr.]
          $\mathbf{\in O(1)}$

    \item [Schleifen]
          $\in$ $i$ Wiederholungen $\boldsymbol{*}$ $O(f)$ teuerste Operation

    \item [Abfolge]
          $O(g)$ nach $O(f)$ $\in O(\boldsymbol{\max} (f;g))$

    \item [Rekursion]
          $\in$ $k$ Aufrufe $\boldsymbol{*}$ $O(f)$ teuerste Operation
  \end{description}
\end{mzImportant}

\begin{mzImportant}
  \index{Mastertheorem}
  \paragraph{Mastertheorem} $a \geq 1$, $b > 1$, $\Theta \geq 0$

  \begin{gather*}
    T(n) = a * T( \frac{n}{b} ) + \Theta (n^k) \\
    \Rightarrow \begin{cases}
      \Theta ( n^k ) \quad          & a < b^k \\
      \Theta ( n^k \log n ) \quad   & a = b^k \\
      \Theta ( n^{\log_b a} ) \quad & a > b^k
    \end{cases}
  \end{gather*}
\end{mzImportant}


\paragraph{Floor/Ceiling} Runden

\begin{description}
  \item [Floor] $\floor{x}$ nach unten
        \index{Floor}

  \item [Ceiling] $\ceil{x}$ nach oben
        \index{Ceiling}
\end{description}
