\chapter{Hashing}

Aus Schlüsseln $S$ werden Adressen/Indices $A$ direkt berechnet,

$$h: S \rightarrow A$$

\begin{description}
  \item [Kollision] $|A| \ll |S| \Rightarrow \neg (h \text{ injekt.})$
        \index{Kollision}

  \item [Synonyme] $h(K_i) = h(K_j)$
        \index{Synonym}

  \item [Kollisionsklasse] $[A]_h = \{ K \in S \mid h(K) = A \}$
        \index{Kollisionsklasse}
\end{description}

\section{Hashfunktionen}

\paragraph{Divisionsrest} $h(K_i) := K_i \mod q$
\index{Divisionsrest-Hash}

\begin{itemize}
  \item $q$ prim $\Rightarrow$ keinen Teiler mit $K$
  \item Optimal bei äquidistanter Schlüsselverteilung
\end{itemize}

\paragraph{Falten} Teilsequenzen des Schlüssels werden addiert (Quersumme) oder XOR-verknüpft (Binär)
\index{Falt-Hash}

\begin{description}
  \item[Rand-Falten] Rechte Teilsequenzen werden gespiegelt
    \index{Rand-Falten}
  \item[Shift-Falten] Teilsequenzen in Reihenfolge
    \index{Shift-Falten}
\end{description}

\paragraph{Mid-Square-Hash} $h(K) := K^2[K.\text{len} - t/2, K.\text{len} + t/2]$
\index{Mid-Square-Hash}

\paragraph{Zufalls-Hash} $K_i$ ist Saat des Zufallsgenerators
\index{Zufalls-Hash}

\paragraph{Ziffernanalyse-Hash} Teilsequenz von $K_i$
\index{Ziffernanalyse-Hash}

\section{Hashtabelle}
\index{Hashtabelle}

\begin{description}
  \item[Kapazität $m$]
  \item[Belegte Adressen $n_a$]
  \item[Belegungsfaktor $\beta = n_a / m$]
    \index{Belegungsfaktor}
    sollte $<.85$ und somit $m > n_a$
  \item[Erfolgreiche Suche] in $S(\beta)$ Schritten
  \item[Erfolglose Suche] in $U(\beta)$ Schritten
\end{description}

\subsection{Kollisionsbehandlung}
\index{Kollisionsbehandlung}

Beim Auftritt einer Kollision $h(K_q) = h(K_p)$ eines gespeicherten $K_q$, welches die Adresse für $K_p$ besetzt:

\paragraph{Sondieren}
\index{Sondieren}
Zusätzliche Klasse Hashfunktionen $h_i$ nach $i$-ter Kollision

\begin{description}
  \item[Linear] $h_i(K_p) = (h_0(K_p) + f(i, h(K_p))) \mod m$
    \index{Lineares Sondieren}

    \begin{itemize}
      \item $S(\beta) \approx \frac{1}{2}(1 + \frac{1}{1 - \beta})$
      \item $U(\beta) \approx \frac{1}{2}(1 + \frac{1}{(1 - \beta)^2})$
    \end{itemize}

  \item[Quadratisch] $h_i(K_p) = (h_0(K_p) + ai + bi^2) \mod m$
    \index{Quadratisches Sondieren}
    \begin{align*}
      h_i(K_p) = & (h_0(K_p) - \ceil{i/2}^2(-1)^i) \\
                 & \mod m
    \end{align*}
    (Sucht in quadratisch wachsenden Abstand in beide Richtungen zur ursprünglichen Adresse)

    \begin{itemize}
      \item Sondierungsfolge versch. Schlüssel korrellieren nicht (Uniform)
      \item $S(\beta) \approx - \frac{1}{\beta} \ln (1 - \beta)$
      \item $U(\beta) \approx \frac{1}{1 - \beta}$
    \end{itemize}


  \item [Zufällig]
        \index{Zufälliges Sonderien}
        Deterministischer Zufallsgenerator generiert Schrittfolge $z_i$
        $$h_i(K_p) = (h_0(K_p) + z_i) \mod m$$

  \item [Double-Hash] Zweite Hashfunktion $h'$
        \index{Double-Hashing}
        \begin{align*}
          h_i(K_p) = & (h_0(K_p) + ih'(K_p)) \\
                     & \mod m
        \end{align*}
\end{description}

Platzhalter für gelöschte Schlüssel zur Signalisierung sondierter Adressen

\paragraph{Verkettung}
\index{Synonymverkettung}
Synonyme werde in dynamischer externen Struktur (Sekundärbereich) in Einfügereihenfolge linear verkettet

\begin{itemize}
  \item $S(\beta) \approx 1 + \frac{\beta}{2}$
  \item $U(\beta) \approx \beta - e^{-\beta}$
\end{itemize}

\section{Hashing auf Externspeicher}

\begin{itemize}
  \item Adresse bezeichnet Bucket der mehrere Daten in Einfügereihenfolge fässt
  \item Überlaufsmethode beliebig, aber Vermeidung langer Sondierungsfolgen, häufig spearater Überlaufsbereich mit dynamischer Zuordnung der Buckets
\end{itemize}

\section{Dynamische Hashstrukturen}

Nachteile der Hashtabelle

\begin{itemize}
  \item Statische Allokationen speicherineffizient
  \item Re-hashing bei Speichererweiterung
\end{itemize}

\paragraph{Erweiterbares Hashing} Digitalbaumk; Bits des Schlüssels oder Hashs steuern Pfad

\paragraph{HAMT: Hashed Array Mapped Tries}
\index{HAMT}
\index{Hashed Array Mapped Tries}
% TODO: TikZ Grafik
Viele Nullzeiger werden durch Bitmap-Kompression vermieden: Knoten mit $n$ Feldern hat $n$ lange Bitmap: 0 zeigt Nullzeiger an, 1 zeigt belegt durch Zeiger

\section{Signaturen}
\index{Signatur}

Möglichst eindeutiges Merkmal eines Datensatzes

\paragraph{Rolling-Hash}
\index{Rolling-Hash}
Signaturhash der mit Hilfe des vorgehenden Fensters (Teilzeichenkette) in konstanter statt linearer Zeit berechnet werden kann